\documentclass[reqno,a4paper,12pt]{amsart}

\usepackage{amsmath,amssymb,amsthm,geometry,xcolor,soul,graphicx}
\usepackage{titlesec}
\usepackage{braket}
\usepackage{xeCJK}
\setCJKmainfont{Kai}
\geometry{left=0.7in, right=0.7in, top=1in, bottom=1in}


\renewcommand{\baselinestretch}{1.3}

\title{固体物理第二次作业}
\author{董建宇 ~~ 2019511017}

\begin{document}

\maketitle
\titleformat{\section}[hang]{\small}{\thesection}{0.8em}{}{}
\titleformat{\subsection}[hang]{\small}{\thesubsection}{0.8em}{}{}


\section{(3.1)}
\begin{enumerate}
	\item (金属的电导率): \\
	假设在$t$时刻电子动量为$\vec{p}(t)$,由Drude模型假设可知,在$t+dt$时刻,电子动量为:
	\[
		\vec{p}(t+dt) = \frac{dt}{\tau} \times 0 + \left( 1-\frac{dt}{\tau} \right) \left( \vec{p}(t) - e\vec{E}\,dt \right).
	\]
	忽略高阶小量,整理可得:
	\[
		\frac{\,d\vec{p}(t)}{\,dt} = -e\vec{E} - \frac{\vec{p}(t)}{\tau}.
	\]
	当达到平衡状态时,有$\frac{\,d\langle \vec{p}(t) \rangle}{\,dt} = 0$,即
	\[
		\langle \vec{v} \rangle = -\frac{e\tau \vec{E}}{m}.
	\]
	由电流密度定义可知:
	\[
		\vec{j} = -ne\langle \vec{v} \rangle = \sigma \vec{E}.
	\]
	则金属的电导率为:
	\[
		\sigma = \frac{n\tau e^2}{m}.
	\]
	其中$n$为电子数密度。
	
	\item (金属的电阻率): \\
	再外加磁场作用下,牛顿第二定律为:
	\[
		\frac{d\vec{p}(t)}{dt} = -e(\vec{E}+\vec{v} \times \vec{B}) - \frac{\vec{p}(t)}{\tau}.
	\]
	当达到平衡状态时,有$\frac{d\langle \vec{p}(t) \rangle}{dt} = 0$,即对于三个电场分量有
	\begin{equation*}
	\begin{aligned}
		E_x =& -\frac{m\langle v_x \rangle}{e\tau} - \langle v_y \rangle B_z \\
		E_y =& -\frac{m\langle v_y \rangle}{e\tau} + \langle v_x \rangle B_z \\
		E_z =& -\frac{m\langle v_z \rangle}{e\tau}
	\end{aligned}
	\end{equation*}
	利用$\vec{E} = \rho \vec{j} = -ne\rho \langle \vec{v} \rangle$,可知矩阵$\rho$为:
	\[
		\rho = \begin{pmatrix}
			\frac{m}{ne^2\tau} & \frac{B_z}{ne} & 0 \\
			-\frac{B_z}{ne} & \frac{m}{ne^2\tau} & 0 \\
			0 & 0 & \frac{m}{ne^2\tau}
		\end{pmatrix}.
	\]
	矩阵的逆为:
	\[
		\sigma = \begin{pmatrix}
			\frac{ne^2\tau m}{m^2+\tau^2e^2B_z^2} & -\frac{ne^3\tau^2B_z}{m^2+\tau^2e^2B_z^2} & 0 \\
			\frac{ne^3\tau^2B_z}{m^2+\tau^2e^2B_z^2} & \frac{ne^2\tau m}{m^2+\tau^2e^2B_z^2} & 0 \\
			0 & 0 & \frac{ne^2\tau}{m}
		\end{pmatrix}.
	\]
	
	\item (霍尔效应): \\
%	在平衡状态下,有洛伦兹力等于电场力,即:
%	\[
%		-e\frac{U_H}{d} = -evB.
%	\]
%	利用电流微观表达式$I = -ned^2v$可知:
%	\[
%		U_H = -\frac{1}{ne}\frac{IB}{d}.
%	\]
	霍尔系数被定义为:
	\[
		H = \frac{\rho_{yx}}{\vert B \vert} = -\frac{1}{ne}.
	\]
	由题意可知,电子数密度为:
	\[
		n = \frac{N}{V} = \frac{N\rho}{m} = \frac{N_A\rho}{M}.
	\]
	则霍尔系数为:
	\[
		H = -\frac{M}{e\rho N_A}.
	\]
	代入数据可知:$H = -2.38\times 10^{-10}m^3/C$. \\
	霍尔电压为:
	\[
		\vert U_H \vert = \frac{HIB}{d} = 4.77\times 10^{-8} V.
	\]
	在测量霍尔电压时,遇到的问题以及解决思路主要有:\\
	霍尔电压极小,不容易被测量到;使用更敏感的电压表。 \\
	会导致发热;搭建一套散热系统。
	
	\item (Drude模型无法解释的金属的性质): \\
	金属中每个电子的热熔不是$\frac{3k_B}{2}$。 \\
	不同金属加热产生电压不同,即不同金属的Seebeck系数不同。 \\
	不同材料的霍尔系数符号不同。 \\
	在计算过程中只考虑价电子而不考虑内层电子。
	
	\item (交流电下电导率矩阵): \\
	当散射平均时间$\tau \to \infty$时,在电场磁场作用下牛顿第二定律为:
	\[
		\frac{d\vec{p}}{dt} = -e\vec{E} - e\vec{v}\times\vec{B}.
	\]
%	平衡状态下有:$\frac{d\vec{p}}{dt} = 0$,即:
%	\[
%		\vec{E} = -\vec{v}\times\vec{B}
%	\]
	利用电流密度围观表达式$\vec{j} = -ne\vec{v} = -\frac{ne\vec{p}}{m_e}$可得:
	\[
		\vec{E} = -\frac{1}{e}\frac{d\vec{p}}{dt} - \vec{v}\times\vec{B} = \frac{m_e}{ne^2}\frac{d\vec{j}}{dt} + \frac{1}{ne}\vec{j}\times\vec{B}.
	\]
	由题意可知:
	\[
		\vec{E} = \vec{E}_0 e^{i\omega t}, ~~ \vec{j} = \vec{j}_0 e^{i\omega t}.
	\]
	代入上式方程可得:
	\[
		\vec{E} = \frac{i\omega m_e}{ne^2}\vec{j} + \frac{1}{ne}\vec{j}\times\vec{B}.
	\]
	则电阻率矩阵为:
	\[
		\rho(\omega) = \begin{pmatrix}
			\frac{i\omega m_e}{ne^2} & \frac{B_z}{ne} & 0 \\
			-\frac{B_z}{ne} & \frac{i\omega m_e}{ne^2} & 0 \\
			0 & 0 & \frac{i\omega m_e}{ne^2} 
		\end{pmatrix}.
	\]
	电导率矩阵为:
	\[
		\sigma(\omega) = \begin{pmatrix}
			\frac{i\omega m_ene^2}{B_z^2e^2-\omega^2m_e^2} & -\frac{ne^3B_z}{B_z^2e^2-\omega^2m_e^2} & 0 \\
			\frac{ne^3B_z}{B_z^2e^2-\omega^2m_e^2} & \frac{i\omega m_ene^2}{B_z^2e^2-\omega^2m_e^2} & 0 \\
			0 & 0 & -\frac{ine^2}{\omega m_e} 
		\end{pmatrix}.
	\]
\end{enumerate}


\section{(3.2)}
\begin{enumerate}
	\item (Drude散射时间): \\
	由第一题可知,金属电阻率为:
	\[
		\rho = \frac{1}{\sigma} = \frac{m_e}{n\tau e^2}.
	\]
	其中,粒子数密度为:
	\[
		n = \frac{N}{V} = \frac{N\rho_m}{m} = \frac{N_A \rho_m}{w}.
	\]
	其中$\rho_m$为金属密度。则有电阻率为:
	\[
		\rho = \frac{m_ew}{\tau e^2N_A\rho_m}.
	\]
	散射时间为:
	\[
		\tau = \frac{m_ew}{e^2N_A\rho_m\rho}.
	\]
	代入表格数据可得银和锂的散射时间分别为(题中电阻率指数应为(-8)):
	\[
		\tau_{Ag} = 3.80 \times 10^{-14}s; ~~ \tau_{Li} = 8.31 \times 10^{-15}s.
	\]
	
	\item (气体动理论): \\
	对于氮气分子,质量约为$m_{N_2} = 28m_p$,其中$m_p$为一个质子质量。气体分子数密度约为:
	\[
		n = \frac{N_A}{V_m} \approx 2.688\times 10^{25}.
	\]
	则散射时间约为:
	\[
		\tau = \frac{1}{n\langle v \rangle\sigma} \approx 1.82 \times 10^{-10}s.
	\]
	氮气分子散射时间远大于金属中电子散射时间,即金属中电子的被散射几率远大于氮气分子的被散射几率,由于金属中自由电子数密度远大于氮气分子数密度。
\end{enumerate}


\section{(3.3)}
\begin{enumerate}
	\item (电阻率): \\
	对于电子而言有:
	\[
		\vec{p}_e(t+dt) = \frac{dt}{\tau_e}\times 0 + \left( 1-\frac{dt}{\tau_e} \right) \left( \vec{p}_e(t) - e\vec{E}\,dt \right)
	\]
	忽略高阶小量,即
	\[
		\frac{d\vec{p}_e(t)}{dt} = -\frac{\vec{p}_e(t)}{\tau_e} - e\vec{E}.
	\]
	平衡状态下有$\frac{d \vec{p}_e(t)}{dt} = 0$,即:
	\[
		\vec{v}_e = \frac{\vec{p}_e}{m_e} = -\frac{e\tau_e \vec{E}}{m_e}.
	\]
	同理,对于自由离子而言有:
	\[
		\vec{p}_i(t+dt) = \frac{dt}{\tau_i}\times 0 + \left( 1-\frac{dt}{\tau_i} \right) \left( \vec{p}_i(t) + e\vec{E}\,dt \right)
	\]
	忽略高阶小量,即
	\[
		\frac{d\vec{p}_i(t)}{dt} = -\frac{\vec{p}_i(t)}{\tau_i} + e\vec{E}.
	\]
	平衡状态下有$\frac{d \vec{p}_i(t)}{dt} = 0$,即:
	\[
		\vec{v}_i = \frac{\vec{p}_i}{m_i} = \frac{e\tau_i \vec{E}}{m_i}.
	\]
	则电流密度为:
	\[
		\vec{j} = -n_ee\vec{v}_e + n_ie\vec{v}_i = \left( \frac{n_ee^2\tau_e}{m_e} + \frac{n_ie^2\tau_i}{m_i} \right)\vec{E} = \sigma \vec{E}.
	\]
	即电阻率为:
	\[
		\rho = \frac{1}{\sigma} = \frac{m_em_i}{e^2(n_e\tau_em_i+n_i\tau_im_e)}.
	\]
	
	\item (热导系数): \\
	由课本公式可知:电子热导系数为:
	\[
		\kappa_e = \frac{4}{\pi} \frac{n_e\tau_ek_B^2T}{m_e}.
	\]
	同理可知,自由离子的热导系数为:
	\[
		\kappa_i = \frac{4}{\pi} \frac{n_i\tau_ik_B^2T}{m_i}.
	\]
	则系统热导系数为:
	\[
		\kappa = \kappa_e + \kappa_i = \frac{4k_B^2T}{\pi}\left( \frac{n_e\tau_e}{m_e} + \frac{n_i\tau_i}{m_i} \right)
	\]
	
	\item (霍尔系数): \\
	由第一题可知:电子与自由离子的电导率矩阵为:
	\[
		\sigma_e = \begin{pmatrix}
			\frac{n_ee^2\tau_em_e}{m_e^2 + \tau_e^2e^2B_z^2} & -\frac{n_ee^3\tau_e^2B_z}{m_e^2 + \tau_e^2e^2B_z^2} & 0 \\
			\frac{n_ee^3\tau_e^2B_z}{m_e^2 + \tau_e^2e^2B_z^2} & \frac{n_ee^2\tau_em_e}{m_e^2 + \tau_e^2e^2B_z^2} & 0 \\
			0 & 0 & \frac{n_ee^2\tau_e}{m_e}
		\end{pmatrix}.
	\]
	\[
		\sigma_i = \begin{pmatrix}
			\frac{n_ie^2\tau_im_i}{m_i^2 + \tau_i^2e^2B_z^2} & \frac{n_ie^3\tau_i^2B_z}{m_i^2 + \tau_i^2e^2B_z^2} & 0 \\
			-\frac{n_ie^3\tau_i^2B_z}{m_i^2 + \tau_i^2e^2B_z^2} & \frac{n_ie^2\tau_im_i}{m_i^2 + \tau_i^2e^2B_z^2} & 0 \\
			0 & 0 & \frac{n_ie^2\tau_i}{m_e}
		\end{pmatrix}
	\]
	总电导率为:
	\[
		\sigma = \sigma_e + \sigma_i = \begin{pmatrix}
			\frac{n_ee^2\tau_em_e}{m_e^2 + \tau_e^2e^2B_z^2} + \frac{n_ie^2\tau_im_i}{m_i^2 + \tau_i^2e^2B_z^2} & \frac{n_ie^3\tau_i^2B_z}{m_i^2 + \tau_i^2e^2B_z^2} - \frac{n_ee^3\tau_e^2B_z}{m_e^2 + \tau_e^2e^2B_z^2} & 0 \\
			\frac{n_ee^3\tau_e^2B_z}{m_e^2 + \tau_e^2e^2B_z^2} - \frac{n_ie^3\tau_i^2B_z}{m_i^2 + \tau_i^2e^2B_z^2} & \frac{n_ee^2\tau_em_e}{m_e^2 + \tau_e^2e^2B_z^2} + \frac{n_ie^2\tau_im_i}{m_i^2 + \tau_i^2e^2B_z^2} & 0 \\
			0 & 0 & \frac{n_ee^2\tau_e}{m_e} + \frac{n_ie^2\tau_i}{m_e}
		\end{pmatrix}
	\]
	令
	\[
		A_e = \frac{m_e}{n_ee^2\tau_e}, ~ A_i = \frac{m_i}{n_ie^2\tau_i}, ~ C_e = -\frac{B_z}{n_ee}, ~ C_i = \frac{B_z}{n_ie}.
	\]
	则总电导率为:
	\[
		\sigma = \begin{pmatrix}
			\frac{A_e(A_i^2 + C_i^2) + A_i(A_e^2 + C_e^2)}{(A_e^2 + C_e^2)(A_i^2 + C_i^2)} & \frac{C_e(A_i^2 + C_i^2) + C_i(A_e^2 + C_e^2)}{(A_e^2 + C_e^2)(A_i^2 + C_i^2)} & 0 \\
			-\frac{C_e(A_i^2 + C_i^2) + C_i(A_e^2 + C_e^2)}{(A_e^2 + C_e^2)(A_i^2 + C_i^2)} & \frac{A_e(A_i^2 + C_i^2) + A_i(A_e^2 + C_e^2)}{(A_e^2 + C_e^2)(A_i^2 + C_i^2)} & 0 \\
			0 & 0 & \frac{A_e + A_i}{A_eA_i}
		\end{pmatrix}.
	\]
	则总电阻率为:
	\[
		\rho = \sigma^{-1} = \begin{pmatrix}
			\frac{A_e(A_i^2 + C_i^2) + A_i(A_e^2 + C_e^2)}{(A_i+A_e)^2 + (C_i+C_e)^2} & \frac{C_e(A_i^2 + C_i^2) + C_i(A_e^2 + C_e^2)}{(A_i+A_e)^2 + (C_i+C_e)^2} & 0 \\
			-\frac{C_e(A_i^2 + C_i^2) + C_i(A_e^2 + C_e^2)}{(A_i+A_e)^2 + (C_i+C_e)^2} & \frac{A_e(A_i^2 + C_i^2) + A_i(A_e^2 + C_e^2)}{(A_i+A_e)^2 + (C_i+C_e)^2} & 0 \\
			0 & 0 & \frac{A_eA_i}{A_e+A_i}
		\end{pmatrix}.
	\]
\end{enumerate}


\end{document}